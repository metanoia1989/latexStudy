
\iffalse
**排版基础: 换行, 分段, 分页**
换行: 自然换行 (若需 强制换行, 可使用 \\ 或 \linebreak)
一般情况下, 不建议使用强制换行

分段: 一个空行或 \par
建议使用空行进行分段 → 简洁直观

分页: 自然分页, 若需 强制分页, 可用 \newpage, \clearpage 或 \pagebreak
一般情况下, 不建议使用强制分页

行间距: 行间距伸展因子 \baselinestretch 或伸展命令 \linespread
\renewcommand{\baselinestretch}{1.2}
\linespread{1.2}
段落间距和段落缩进:用自动设定的即可, 英文每节的第一段首行不会自动缩进

-----------------------

**排版基础: 长度**  
\setlength{长度数据命令}{长度}
\addtolength{长度数据命令}{长度}

常用长度数据命令 (变量)
- \textwidth, \textheight, \parindent, \parskip, \baselineskip
? ?
1 \setlength{\textwidth}{15cm}
2 \setlength{\textheight}{21cm}
3 \setlength{\hoffset}{-5mm} % 长度有时也可以是负值


-----------------------


**排版基础: 水平间距**      
◦ \quad → 产生一段宽度为 1em 的水平空白
◦ \qquad → \quad 的两倍
◦ \, → 大约为 \quad 的 3/18
◦ \hspace{宽度} → 产生指定宽度的水平空白
◦ \hspace*{宽度} → 若要在行首产生一定的空白, 则需使用此命令
◦ \hfill → 根据排版需要插入空白, 撑满整行
◦ \hphantom{文本内容}: 水平空白的宽度等于文本内容的总宽度

------------------------

**排版基础: 垂直间距**      
◦ \smallskip → 垂直空白高度为 3pt plus 1pt minus 1pt
◦ \medskip → \smallskip 的两倍
◦ \bigskip → \smallskip 的四倍
◦ \vspace{高度} → 产生指定高度的垂直空白
◦ \vspace*{高度}→ 同 \vspace, 主要同在页面的顶部
◦ \vphantom{文本内容} → 垂直空白的高度等于文本内容的总高度

\fi

\documentclass{article}
\usepackage[UTF8]{ctex}

\usepackage[breaklinks,colorlinks,linkcolor=black,citecolor=black,urlcolor=black]{hyperref}

\renewcommand{\baselinestretch}{1.2}
\linespread{1.9}

\setlength{\textwidth}{15cm}
\setlength{\textheight}{21cm}
\setlength{\hoffset}{-5mm}

% 页边距
\usepackage{geometry}
\newgeometry{left = 3 cm, right = 3 cm, top=2.5cm, bottom=1.5cm}


\title{排版基础}
\author{AdamSmith}
\date{\today}

% 书签
\usepackage{hyperref}  % 在\begin{document}之前加入
\hypersetup{CJKbookmarks=true}

\begin{document}
    \maketitle % 打印标题
    
    \tableofcontents % 目录

    \section{换行}

    你好哈哈    \\
    你好哈哈    
    
    \section{段落}

    佛告波斯匿王:“如是,大王。如是,大王,世少有人得胜妙财利能不贪著,不起放逸,不起邪行。世多有人得胜妙财利,于财放逸,而起贪著,起诸邪行。大王当知,彼诸世人得胜财利,于财放逸,而起贪著,作邪行者,是愚痴人,长夜当得不饶益苦。大王,譬如猎师、猎师弟子,空野林中张网施羂,多杀禽兽,困苦众生,恶业增广。如是,世人得胜妙财利,于财放逸,而起贪著,造诸邪行,亦复如是。是愚痴人,长夜当得不饶益……”
    \par 哈哈\par
    波斯匿王白佛:“大富,世尊,钱财甚多,百千巨亿金钱宝物,况复余财!世尊,彼摩诃男在世之时,粗衣恶食,如上广说:”


    \section{分页}
    换页了,去下一页查看吧
    \pagebreak

    时,波斯匿王来诣佛所,稽首佛足,退坐一面,白佛言:“世尊,此舍卫国有长者,名摩诃男,多财巨富,藏积真金至百千亿,况复余财!世尊,摩诃男长者如是巨富,作如是食用:食粗碎米,食豆羹,食腐败姜,著粗布衣、单皮革屣,乘羸败车,戴树叶盖,未曾闻其供养施与沙门、婆罗门,给恤贫苦、行路顿乏、诸乞丐者;闭门而食,莫令沙门、婆罗门、贫穷、行路、诸乞丐者见之。”
    
    We provide some predefined CSS class names to provide access for developers to style layout of a page globally in Docusaurus. The purpose is to have stable classnames shared by all themes that are meant to be targeted by custom CSS.
    
    \section{水平间距}

    AB\quad{}CD\qquad{}EF\,哈哈
    金刚葫芦娃\hspace{18pt}英雄\\
    \hspace*{20pt}首行空白使用

    \hfill{}是他就是他,我们的英雄小娜扎。这个是自动撑满的。

    \hphantom{呼和浩特}插入了呼和浩特四字宽度的空白
    
    
    \section{垂直间距}
    
    純陀,一些比丘超越了所有的無所有處,他們進入了非想非非想處。這時他們可能
會認為自己安住在漸損之中。純陀,在聖者的律之中,這不稱為漸損;在聖者的律之中,
這稱為安住在寂靜之中。

    \smallskip

    純陀,在善法之中即使生起一個心念也能帶來很大的利益,更遑論跟隨善心而作出
身行和口行了

    \medskip{}

    他人會緊緊取著世俗的見,不易放捨,但我會放捨見取使自己上昇。
    
    \vspace{15pt}

    賢友們,什麼是不善?什麼是不善的根源?什麼是善?什麼是善的根源呢?殺生是
不善的,偷盜是不善的,邪淫是不善的,妄語是不善的,兩舌是不善的,惡口是不善的,
綺語是不善的,貪欲是不善的,瞋恚是不善的,邪見是不善的。這就是稱為不善了。

    \vphantom{賢友們,什麼是善呢?不殺生是善的,不偷盜是善的,不邪淫是善的,不妄語是善
    的,不兩舌是善的,不惡口是善的,不綺語是善的,不貪欲是善的,不瞋恚是善的,正見
    是善的。這就是稱為善了。}

    賢友們,什麼是善呢?不殺生是善的,不偷盜是善的,不邪淫是善的,不妄語是善
的,不兩舌是善的,不惡口是善的,不綺語是善的,不貪欲是善的,不瞋恚是善的,正見
是善的。這就是稱為善了。


    \section{特殊字符显示}
    
    \textbf{有 10 个字符被赋予了特殊用途, 需要使用相应的命令才能输出}

    \# 
    \$ 
    \% 
    \{ 
    \} 
    \~{} 
    \_{} 
    \^{} 
    \& 
    \textbackslash{}
    
    \textbf{符号 “>”, “<”, “|” 被定义成数学符号, 只能用在数学模式中, 若要在普通文本中输出, 需使用相应的命令}

    |<> % 实际上是可以输出的

    \textbar
    \textless
    \textgreater
    
    
    - -- ---

\end{document}