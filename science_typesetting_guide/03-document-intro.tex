% 文档和导言区


\iffalse
离谱啊,多行注释竟然这样搞,不够语义化啊    

\documentclass[选项]{文档类}
a 位于源文件的最前面, 用于指定文档的整体结构和布局, 必须且只能选一种
◦ 常用 文档类: article, book, beamer,
ctexart, ctexbook, ctexbeamer
◦ 常用 选项:
- 10pt(缺省值), 11pt, 12pt → 指定基本字体的大小
- letterpaper(缺省值), a4paper, a5paper, ... → 指定纸张的大小
- 单双面选项: oneside, twoside, openright, openany
- 数学公式: leqno (公式编号在左边), fleqn (靠左显示行间公式)

导言区: \documentclass 和 \begin{document} 之间的区域
◦ 导言区用于放置 全局控制命令, 如: 调用宏包, 设置页面大小, ...
◦ 放在导言区的命令对整个文档都起作用


宏包使用
宏包调用方法 (只能出现在导言区)
\usepackage[选项]{宏包名}
◦ 如果宏包不带选项, 则可以多个一起调用, 如:
1 \usepackage{amsmath,amssymb,amsfonts}
2 \usepackage[pagebackref]{hyperref}
3 \usepackage[numbers,sort&compress]{natbib}

◦ 常用宏包:
- geometry, fancyhdr, natbib, float, caption
- amsmath, amssymb, amsfonts, amsthm, ntheorem, bm, mathtoos
- xcolor, graphicx, subfigure, epstopdf
- longtable, colortbl, tcolorbox, mdframed
- algorithm, algpseudocode, listings

\fi
\documentclass[18pt,a4paper]{ctexart}

\usepackage{amsmath} % AMS 数学公式 宏包
\usepackage{amssymb} % AMS 数学公式 宏包
\usepackage{amsfonts} % AMS 数学字体 宏包
\usepackage{graphicx} % 插图 宏包
\usepackage{xcolor} % 彩色 宏包


\begin{document}
    The Euler equation is given by 
    $$ e^{ix} \triangleq \cos(x) + i\sin(x). $$
    
    CTEX 宏集中提供了三个中文文档类: ctexart, ctexbook, ctexbeamer

    欧拉公式是

    $$ e^{ix} \triangleq \cos(x) + i\sin(x). $$
\end{document}